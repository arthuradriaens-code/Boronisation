\documentclass{article}
\usepackage{amsmath}
\title{W7-X relevant TOMAS wall conditioning experiments}
\author{Arthur Adriaens}

\begin{document}
\maketitle

\section{Short overview}
\begin{center}
    \addtolength{\leftskip} {-2cm} % increase (absolute) value if needed
    \addtolength{\rightskip}{-2cm}
    \begin{tabular}{||c p{4cm} p{4cm} c c||}
         \hline
         Amount & kind of samples & need & project & TOMAS days \\ [0.5ex]
         \hline\hline
         28 & boronized graphite & H \& He and H+He GD & W7-X & 6 \\
         20 & hydrogen doped boronized graphite & He ICWC & W7-X & 4 \\
         \hline
    \end{tabular}
\end{center}
\section{Scientific relevance and reasoning}
W7-X employs weekly Glow Discharge (GD) in H$_2$ and daily in helium for
impurity and hydrogen removal respectively \cite{AndreiNucMatEn}. and Electron
Cyclotron Wall Conditioning (ECWC) in H and He plasma (in the form of pulse trains)
is applied for density control in hydrogen plasmas during experimental days, 
its effects have been reported (\cite{Goriaev_2020},\cite{WAUTERS2018235} and
EUROFUSION WPS1-PR(16) 16175). Eventually, W7-X will leverage the ICRF
antenna to perform ICWC.

\subsection*{boronized graphite}
The interaction of the hot plasmas will erode the first (boronized) wall, it
would be interesting to be able to predict to what extend, to this end we will
measure ion energy distributions for W7-X relevant GD settings using the
Retarding Field Energy Analyzer (RFEA) and expose samples for chosen
environments. The aim is to then predict with an erosion code like ERO or
rustBCA the observed erosion. If this is possible, it would showcase the
effectiveness of the erosion codes, enabling it's deployment on W7-X
measured/calculated wall fluxes.

\subsection*{hydrogen doped boronized graphite}
W7-X has diverted EC plasma, as such only the divertor area
is conditioned by ECWC which erodes the boron layer within seconds.
As such it is not relevant to study ECWC on TOMAS, and we will focus on
ICWC.
% which may provide a more efficient cleanup \cite{WEGAIC}. As preparations for
%the possible usage of ICWC in W7-X 
We wish to observe a time law for de-trapping to interpolate to W7-X.
\section{TOMAS setup and days estimate}
\subsection*{GD}
The GD settings used at TOMAS will mimick the ones used in W7-X, namely 1.5A
$\approx300$V at $\approx4.5\times10^{-3}$mbar  for H$_2$ and 1.0A $\approx
200$V at $\approx 3.5\times10^{-3}$mbar for He\cite{AndreiNucMatEn}.  As the
current and voltage relate via ohms law it might not be possible to exactly
reproduce both the current and voltage. As classically sputtering is linear
with flux (or current) and a power law with energy, matching the voltage is 
of higher importance.

GD experiments will be carried out on 28 boronized graphite samples, grouped 4
per shot for 7 shots. 4 samples per exposures to have reasonable statistics and
4 exposures for different energies each to have reasonable confidence in the
yield-energy curve form.  Furthermore 4 samples per exposure for 3 exposures
will be conducted for yield-time estimation (should be linear, extra
measurement for validation).
\subsection*{EC}
EC experiments will be carried out on 20 hydrogen-doped boronized graphite
samples, grouped 4 per shot for 6 shots to have sufficient statistics
in outgassing per time and in the observation of the time law.
\subsection*{Both}
The days estimate comes from the possibility to do 2 exposures per day,
with one day of preparatory work, calculated on a maximum of 3 exposure days per week.
During experiments the QMS, the MW interferometer and optical spectroscopy will
be acquiring data.
\section{Sample analysis}
Most of this will be outsourced to experts within FZJ whom will
be credited appropriately. 
On the non-doped samples mainly erosion will be measured while on the doped samples outgassing
will be measured. On the erosion samples other measurements will also be performed
\begin{enumerate}
    \item roughness measurement before and after
    \item (if adequate) ellipsometry before and after
    \item FIBSEM after and on one control
\end{enumerate}
\subsection{Roughness measurements}
The exposure to the plasma may give inhomogeneous sputtering which might be
of interest, as such a simple 1D profilometry measurement may be made before
and after exposure, or, if needed, a full 2D surface map. Both of the needed
devices are located in the mirror lab overseen by dr. Litnovsky.
\subsection{Thickness estimate}
Thickness estimates may be made using ellipsometry on the non-doped specimens prior
to the exposure and after exposure prior to being analysed using FIBSEM (dr.Rasinski).
\bibliographystyle{plain}
\bibliography{sources}
\end{document}
