\documentclass{article}
\usepackage{amsmath}
\title{W7-X relevant TOMAS wall conditioning experiments}
\author{Arthur Adriaens}

\begin{document}
\maketitle

\section{Short overview}
\begin{center}
    \addtolength{\leftskip} {-2cm} % increase (absolute) value if needed
    \addtolength{\rightskip}{-2cm}
    \begin{tabular}{||c p{4cm} p{4cm} c c||}
         \hline
         Amount & kind of samples & need & project & TOMAS days \\ [0.5ex]
         \hline\hline
         22 & boronized graphite & H \& He and H+He GD & W7-X & 6 \\
         24 & hydrogen doped boronized graphite & H \& H+He ICWC & W7-X & 9 \\
         \hline
    \end{tabular}
\end{center}
The boronized graphite (non-doped) is ideally made as soon as possible,
instructions on when the hydrogen doped boronized graphite should be made will
follow.
\section{Scientific relevance and reasoning}
W7-X employs weekly Glow Discharge (GD) in H$_2$ and daily in helium for
impurity and hydrogen removal respectively \cite{AndreiNucMatEn}. and Electron
Cyclotron Wall Conditioning (ECWC) in H and He plasma (in the form of pulse trains)
is applied for density control in hydrogen plasmas during experimental days, 
its effects have been reported (\cite{Goriaev_2020},\cite{WAUTERS2018235} and
EUROFUSION WPS1-PR(16) 16175). Eventually, W7-X will leverage the ICRF
antenna to perform ICWC.

\subsection*{boronized graphite}
The interaction of the hot plasmas will erode the first (boronized) wall, it
would be interesting to be able to predict to what extend, to this end we will
measure ion energy distributions for W7-X relevant GD settings using the
Retarding Field Energy Analyzer (RFEA) and expose samples for chosen
environments. The aim is to then predict with an erosion code like ERO or
rustBCA the observed erosion. If this is possible, it would showcase the
effectiveness of the erosion codes, enabling it's deployment on W7-X
measured/calculated wall fluxes.

\subsection*{hydrogen doped boronized graphite}
W7-X has diverted EC plasma, this means that the divertor area is the only part
conditioned by ECWC which erodes the present boron layer within seconds.  As
such it is not relevant to study ECWC on TOMAS, and we will focus on ICWC.  The
ICWC settings used for the exposures at TOMAS cannot directly mimick W7-X,
TOMAS lacks in power and magnetic field strength and as such doesn't have the
same heating fundamentals and wall conditioning scenario.  However it is
possible to compare the two, mainly by predicting the de-trapping rate
according to relevant pde's, as e.g mentioned in McCracken\cite{McCRACKEN}.
Observing these kinds of rates in W7-X relevant ICWC plasmas at TOMAS would
make it possible to interpolate to W7-X itself given the right amount of insight.
\section{TOMAS setup and days estimate}
\subsection*{GDWC}
The GD settings used at TOMAS will mimick the ones used in W7-X, namely 1.5A
$\approx300$V at $\approx4.5\times10^{-3}$mbar  for H$_2$ and 1.0A $\approx
200$V at $\approx 3.5\times10^{-3}$mbar for He\cite{AndreiNucMatEn} and one with
both H and He.  As the
current and voltage relate via ohms law it might not be possible to exactly
reproduce both the current and voltage. As sputtering rates are predicted to be linear
with flux (or current) but behave as a power law with energy, matching the voltage is 
of higher importance.

There are thus three main scenarios, for each of these we will expose 6 samples
and 2 dummy (non-coated) samples. Additionally 4 samples will serve as control
samples, which won't be exposed to see if atmospheric effects deteriorate the
sample. In total we thus require 22 samples.  Each exposure will take one day
with a preperatory day and as such this will take 6 days in TOMAS time.
\subsection*{ICWC}
The ICWC settings at TOMAS will mimick W7-X where it can, namely the frequency
of the ICRF antenna will be set to 38MHz, keeping the pressure at the sample
manipulator as as close to the ones used in W7-X's wall as possible. The gasses
used will be once helium and once helium+hydrogen. To interpolate power
differences need to be observed as well as time traces. As such with two gasses 
each having one time trace at maximum power of at least 4 exposures and one energy
trace of at least 4 exposures, each hosting 3 samples (to have a simple error estimate)
the total amount of samples comes to $2\times 2\times 4\times 3=24$ samples which
should take 9 days to expose.
\subsection*{How the days are estimated}
The days estimate comes from the possibility to do 2 exposures per day,
with one day of preparatory work, calculated on a maximum of 3 exposure days per week.
During experiments the QMS, the MW interferometer and optical spectroscopy will
be acquiring data.
\section{Sample analysis}
\subsection*{GD erosion rates}
As it might be interesting in the erosion predictions, surface roughness will also be measured
(as it may affect the rapidity), in total the steps are:
\begin{enumerate}
    \item roughness measurement before exposure (using a profilometer at the mirror lab)
    \item (if adequate) ellipsometry before exposure (mirror lab)
    \item exposure to relevant conditions
    \item rougness measurement after exposure (using a profilometer at the mirror lab)
    \item (if adequate) ellipsometry after exposure (mirror lab)
    \item FIBSEM after exposure (dr. Rasinski)
\end{enumerate}
\subsection*{De-trapping amount}
Dr. Houben may be able to measure the amount of trapped hydrogen before and after exposure using
thermal desorption, if this is not the case, or if it may be less accurate, IBA should probably be used.
\bibliographystyle{plain}
\bibliography{sources}
\end{document}
