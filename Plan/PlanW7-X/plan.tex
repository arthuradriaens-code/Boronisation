\documentclass{article}
\usepackage{amsmath}
\title{W7-X relevant TOMAS wall conditioning experiments}
\author{Arthur Adriaens}

\begin{document}
\maketitle

\section{Short overview}
\begin{center}
    \addtolength{\leftskip} {-2cm} % increase (absolute) value if needed
    \addtolength{\rightskip}{-2cm}
    \begin{tabular}{||c p{4cm} p{4cm} c c||}
         \hline
         Amount & kind of samples & need & project & TOMAS days \\ [0.5ex]
         \hline\hline
         22 & boronized graphite & H \& He and H+He GD & W7-X & 6 \\
         24 & hydrogen doped boronized graphite & H \& H+He ICWC & W7-X & 8 \\
         \hline
    \end{tabular}
\end{center}
The boronized graphite (non-doped) is ideally made as soon as possible,
instructions on when the hydrogen doped boronized graphite should be made will
follow.
\section{Scientific relevance and reasoning}
W7-X  employs Glow Discharge Cleaning in hydrogen and helium in the initial
wall conditioning  stage after the vacuum vessel venting. It is also conducted
weekly in hydrogen to further improve the wall conditions and daily in helium
to desaturate the walls from hydrogen\cite{AndreiNucMatEn} unless the
hydrogen/helium ratio is desired to be close to 1 in the constitutive
experiments. The GDC cleaning is done only before the first boronization. After
the boronization it is suspended to avoid a possible boron layer erosion.
However, it is not scientifically proven. ECWC in the form of pulse trains is
used in helium or hydrogen to establish the density control throughout
experimental days (\cite{Goriaev_2020},\cite{WAUTERS2018235} and
EUROFUSION WPS1-PR(16) 16175). The installed ICRH antenna provides an opportunity to
employ ICWC, which should be more effective than ECWC, and, moreover can be used when
ECWC is not operable, for example at low (1.7/1.8 T) magnetic field operation.
Since the mechanism of ICWC for W7-X is supposed to rely on moderate fluxes of
charge-exchange neutrals, it can also contribute to the erosion of the boron
layers in the main PFCs.

\subsection*{boronized graphite}
As mentioned GD is avoided except prior to the  first boronisation as it is
thought to erode the wall, it would be interesting to be able to predict to
what extend, to this end we will measure ion energy distributions for W7-X
relevant GD settings using the Retarding Field Energy Analyzer (RFEA) and
expose samples for chosen environments.  As the GD system on tomas and the one
on W7-X are the same with just a smaller volume, measurements obtained on TOMAS
directly predicts what will be observed at W7-X.

As we have a RFEA and ToF-NPA, we also aim to predict with an erosion code like
ERO or rustBCA the (possibly) observed erosion. 
\subsection*{hydrogen doped boronized graphite}
W7-X has diverted EC plasma, this means that the divertor area is the only part
conditioned by ECWC which erodes the present boron layer within seconds.  As
such it is not relevant to study ECWC on TOMAS, and we will focus on ICWC.  The
ICWC settings used for the exposures at TOMAS, due to low confinement, may
mimick at TOMAS's sample manipulator, the charge-exchange flux at the edge of
W7-X.  We will also attempt to predict de-trapping rates using rustBCA.
\section{TOMAS setup and days estimate}
\subsection*{GDWC}
The GD settings used at TOMAS will mimick the ones used in W7-X, namely 1.5A
$\approx300$V at $\approx4.5\times10^{-3}$mbar  for H$_2$ and 1.0A $\approx
200$V at $\approx 3.5\times10^{-3}$mbar for He\cite{AndreiNucMatEn} and one with
both H and He.  
There are thus three main scenarios, for each of these we will expose 6 samples
and 2 dummy (non-coated) samples. Additionally 4 samples will serve as control
samples, which won't be exposed to see if atmospheric effects deteriorate the
sample. In total we thus require 22 samples.  Each exposure will take one day
with a preperatory day and as such this will take 6 days in TOMAS time.
\subsection*{ICWC}
The ICWC settings at TOMAS will mimick W7-X where it can, namely the
frequencies of the ICRF antenna will be 25MHz and 38MHz, keeping the pressure
at the sample manipulator as close to the ones used in W7-X's wall as possible.
The gasses used will be once helium and once helium+hydrogen. As such with two
gasses each having two frequencies each hosting 6 samples (to have good
statistical accuracy) the total amount of samples comes to $2\times 2\times
6=24$ samples which should take 8 days to expose.
\subsection*{How the days are estimated}
The days estimate comes from the possibility to do 1 exposures per day,
with one day of preparatory work, calculated on a maximum of 3 exposures per week.
During experiments the QMS, the MW interferometer and optical spectroscopy will
be acquiring data.
\section{Sample analysis}
\subsection*{GD erosion rates}
As it might be interesting in the erosion predictions, surface roughness will also be measured
(as it may affect the rapidity), in total the steps are:
\begin{enumerate}
    \item roughness measurement before exposure (using a profilometer at the mirror lab)
    \item (if adequate) ellipsometry before exposure (mirror lab)
    \item exposure to relevant conditions
    \item rougness measurement after exposure (using a profilometer at the mirror lab)
    \item (if adequate) ellipsometry after exposure (mirror lab)
    \item FIBSEM after exposure (dr. Rasinski)
\end{enumerate}
\subsection*{De-trapping amount}
Dr. Houben may be able to measure the amount of trapped hydrogen before and
after exposure using thermal desorption, if this is not the case, or if it may
be less accurate, IBA methods should probably be used.
\bibliographystyle{plain}
\bibliography{sources}
\end{document}
