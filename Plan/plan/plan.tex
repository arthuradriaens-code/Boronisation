\documentclass{article}
\usepackage{amsmath}
\title{Research plan}
\author{Arthur Adriaens}

\begin{document}
\maketitle

\section{Thickness and density estimation}
We wish to know how the various wall conditioning schemes affect the PFM, to
this end we need to know both the sample thickness before and after the
exposures, as well as it's density.  We want to know the density as to see how
it is dependent on the substrate and to infer how many atoms got sputtered.
\subsection{Boronized Tungsten}
My proposal for measuring the thickness
of the samples would be to use three methods, the first two methods concern a
general estimate of the thickness whilst the third would be sample-specific.
We first half-coat one sample by applying sticky tape during coating and
peeling it off afterwards, enabling the first method: the use of a profilometer
to determine the jump in height. The second method which we'll also apply on
this half-coated sample is the use of a SEM either by drilling small holes in
the coated surface and determine the thickness that way or looking at the side
of the sample.  The third method is to use the ellipsometer in the mirror lab,
this last method is also what we'll be using for the normal samples, it is
however good to have many thickness estimates of the half-coated sample as
we'll need it later and as it's a good check to see if the ellipsometry
measurements are accurate.  Note that most of these techniques (especially the
ellipsometry) will only be possible if the tungsten surface is polished to within
a few nanometers.
 \vspace{0.2cm} \\
To now measure the density we'll perform ion beam analysis on the half coated
sample. To see how this works, please allow the following train of thought: After
analysing the experimental data from the IBA (ERDA) we'd know that there are x amount
of $B^{10}$ atoms per cm$^2$ and y amount of $B^{11}$ atoms per cm$^2$ on the
surface of the sample, using the mass of both of these isotopes and the
measured thickness (of which we have 3 independent measurements), we can infer
the density:
\begin{equation}
    \rho = \frac{1}{d \times 1 cm^2} \left( x \times \text{massa }B^{10}   \text{ isotope} + y \times \text{massa }B^{11}   \text{ isotope}\right)
\end{equation}
Assuming this density to hold for the other samples, we can infer how many atoms 
were sputtered from the change in thickness which we'll measure using ellipsometry.
\subsection{Boronized Graphite}
My proposal for measuring the thickness of the boron layer on graphite would be
to use two methods, we'll also be using the half-coated sample enabling the use
of a profilometer. The second method which we'll apply in conjunction is by
using the ellipsometer in the mirror lab, note that the SEM wouldn't be able to
see the difference between carbon and boron and is thus not possible to use
here. We then proceed as mentioned for tungsten, inferring the density using
IBA.
\section{Doping estimation}
As we'd later want to measure the outgassing efficiency of the various wall conditioning 
systems, we'd like to dope the samples with some atoms (e.g deuterium) and measure the concentration
before and after the various wall conditioning schemes.
We'll probably measure these concentrations using ERDA.
\section{Exposure: Erosion rate}
We'd like to expose the samples to either hydrogen, helium or mixed hydrogen
and helium. Each under different wall conditioning regimes, either Glow Discharge,
Ion Cyclotron Wall Conditioning, Electron Cyclotron Wall Conditioning or ICWC and ECWC
at the same time. We'll first do some spectroscopy to see the amount of impurities in
our plasma (if any are visible, our spectrometer is quite low-resolution). And then
expose the pure boron to test the erosion rate with different powers.
\subsection{ICWC}
In TOMAS the IC creates both neutrals and ions, mostly with energies below
1keV.  Unfortunately the TOMAS IC antenna system only enables coupling to the
plasma up to 40MHz, as such this will be our operating frequency together with
a magnetic field of 0.114T (2000A input current) at 1500W, 3500W and 5500W
injected power, with a pressure of $10^{-4}$ mbar during the discharge of the
1500W, maintining the same base pressure (not the neutrals pressure) for the
higher powers. I.e when the penning gauge indicates $10^{-3}$ mbar without IC
and $10^{-4}$ mbar with IC at 1500W, we'll be doing 5500W for the same
$10^{-3}$ mbar gas. Whilst the powers we'll use are very small compared to
larger devices, due to the way everything is measured, as will be mentioned, it
might be possible to extrapolate to bigger devices.  ICWC at TOMAS is a
monopole working in a mode conversion scheme, as such most of the particles
evenly spread to the wall.  This enables us to extrapolate measurements
performed by the ToF-NPA to the full vessel and use it as a prediction on how
neutrals erode the sample, this already has been implemented in the NPA
analysis code making it very straightforward to infer expected erosion rates.
The question now is the erosion rates due to ions, unfortunately the retarding
field energy analyser (RFEA) seems to have some hickups for the moment when
doing IC discharges which we'll try to fix over the next couple of weeks.  If
it's not properly fixed when experiments are carried out, we will have to rely
on simulations to determine the amount of ions and their energies in the
plasma.
\subsection{ECWC}
Efficiency of ECWC for fuel removal has been less investigated in current
devices but as it will be used for conditioning in JT-60SA and W7-X, there is
an upsurge in interest. We don't expect any 10eV$<$ energy neutrals, as such
the main player of outgassing and erosion will be ions.  As mentioned we either
need some kind of RFEA measurements to estimate the erosion rate or somehow
correlate the electron density and temperature to the ion density and
temperature (as no neutrals will be observed).
\subsection{ECWC+ICWC}
These sometimes are used in conjunction, so we'll also do exposures with these,
note that the erosion rate will drastically increase. The max combined power
should be around 11kW of injected power, we may scale as: 1.5kW EC + 1.5kW IC,
3.5kW EC + 3.5kW IC, 5.5kW EC + 5.5kW IC, or total = 3kW, 7kW and 11kW of
power.  I'd do the combined erosion after the IC and EC as we'll have a more
concrete idea of what to expect.
\subsection{Glow Discharge}
Even though GD is falling out of favour, we'll do some exposures if time permits, with the same 
pre-measurements as IC.
\section{Extrapolation to larger devices}
If we can simulate the particle flux, using e.g a modified version of the tomator code, 
we might be able to verify the erosion prediction this way and have confidence that this
kind of code may work on bigger devices.

It is also possible that some mechanisms are dominant and easily scalable, for
example, it might be that the IC sputtering is mainly caused by the RF sheath
formed in front of the antenna.  The RF sheath induced sputtering may scale
logaritmically with power, if this is also the case with the sample erosion, we
can be confident that it's the main sputtering candidate and thus easily
scalable to other antenna's (by simulating their sheath).

\section{Exposure: Outgassing efficiency}
This is for much later, after all the erosion estimations have been done, but
under all the previously mentioned techniques, we'll use the same measurements
and estimate the outgassing rate by constructing a sample with the same amount
of doping in the BCA simulation.
\end{document}
