\documentclass{article}
\usepackage{amsmath}
\title{Boronisation research plan}
\author{Arthur Adriaens}

\begin{document}
\maketitle

\section{Short overview}
\begin{center}
    \addtolength{\leftskip} {-2cm} % increase (absolute) value if needed
    \addtolength{\rightskip}{-2cm}
    \begin{tabular}{||c p{4cm} p{4cm} c c||}
         \hline
         Amount & kind of samples & need & project & TOMAS days \\ [0.5ex]
         \hline\hline
         12 & hydrogen doped graphite & Test ICWC vs ECWC & W7-X & 18 \\
         12 & hydrogen doped boron-coated graphite & & & \\
         \hline
         56 & Boronized tungsten, doped with deuterium & Test efficiency of IC and GD & ITER & $\approx$35\\
         \hline
    \end{tabular}
\end{center}
Note that these samples don't include the five ITER-contract mentioned 
"test depositions and characterization of the layers,...to make sure D content can be produced reliably."
\section{Scientific relevance and reasoning}
\subsection*{W7-X}
Glow discharge in H$_2$ plasma and ECWC in He plasma have been actively used on
W7-X to condition the wall and it's effects have been reported
(\cite{Goriaev_2020},\cite{WAUTERS2018235} and EUROFUSION WPS1-PR(16) 16175)
while it is possible for stellarators to have homogeneous EC wall conditioning,
ICWC may provide a more efficient cleanup \cite{WEGAIC}. As preparations for
the possible usage of ICWC in W7-X it should be investigated how ECWC and ICWC
compare both in removal from pure graphite and from boron-coated graphite (as
both are possible PFMs in W7-X).  To this end baseline ECWC experiments need to
be carried out on TOMAS, as well as comparative ICWC experiments, both in W7-X relevant plasmas.

As mentioned in EUROFUSION WPS1-PR(16) 16175, ECWC outgassing follows a time
power law ($\propto t^{-0.7}$), a similar law for removal needs to be observed
on TOMAS to show that it is equipped to compare IC and EC, after which similar
experiment needs to be carried out to determine the IC time power law and scale
difference between EC and IC on TOMAS.

To obtain W7-X relevant conditions in TOMAS, the IC frequency will be set to
38MHz, mimicking the W7-X minority heating and the second harmonic heating
scenarios (which we of course don't have due to the low magnetic field).  Both
the electron density, temperature and ideally ion energy distribution 
will be kept as close as possible to the one observed at W7-X 
(reflectometer + langmuir probe measurements + modelling on W7-X, LP
, interferometer and RFEA on TOMAS), the plasma species will be Helium. 

During experiments the QMS, the MW interferometer and optical spectroscopy will
be acquiring data.
\subsection*{ITER}
Following the ITER contract [ADCYGE] the 2 ITER available wall
conditioning schemes GDC and ICWC will be tested on deuterium-doped 
boronized tungsten. To this end ITER relevant plasmas need to be created
in the TOMAS machine on the sample holder position, after which the necessary 
exposures are carried out.
To this end NPA measurements need to be carried out to find a fluence relevant to a 
20-minute ICWC procedure at 10-20\% duty cycle (as layed out in the contract).
During experiments the QMS, the MW interferometer and optical spectroscopy will
be acquiring data.
\section{TOMAS setup and days estimate}
\subsection{W7-X}
The exposures will all keep the previously mentioned plasma characteristicx to make a
time-removal relation possible.  As W7-X boronizes their walls (monthly), EC
and IC experiments will be carried out on 24 hydrogen-doped boronized graphite
samples, grouped 4 per shot 3 per WC scheme (2: EC and IC).  
This should give insight into the de-trapping efficiency of boron in W7-X.
As the rapidity of boron erosion under plasma operation is still an open
question, this will also be measured post-mortem.
\subsection{ITER}
ICWC: 8*4 = 32 samples (8 per exposure, 4 exposures out of which 2 are test exposures and 2 are final, 
performed at 70$^\circ$C and 240$^\circ$C).

GDC: 8*3 = 24 samples (8 per exposure, 3 exposures out of which 1 is a test and 2 are final,
one at 70$^\circ$C and one at 240$^\circ$C).

TOMAS time estimate: the ITER plan gives a maximum of 2 months, this is reasonable,
a full exposure of 8 samples may take up to 5 days (3 day of preparation such as finding the 
ideal setting and cleaning the wall, 1 day of pumping down and 1 day of exposure)
as there are 7 such exposures we estimate 35 days of experiments.
\section{Sample analysis}
Most of this will probably be outsourced to experts within FZJ whom will
be credited appropriately. The primary goal on all exposed samples
is to measure the fuel removal, nevertheless it is possible to extract additional
information like roughness before and after exposure and erosion of the boron layer.
As such the plan is: 
\begin{enumerate}
    \item sample preparation
    \item roughness measurement
    \item (if adequate)ellipsometry
    \item TOMAS exsposure
    \item roughness measurement
    \item De-trapping measurements
    \item (if adequate) ellipsometry 
    \item FIBSEM
\end{enumerate}
Here will follow a short overview of the analysis methods.
\subsection{fuel removal estimates}
The elemental concentrations need to be inferred for the samples, this will be accomplished
using Thermal Desorption Spectroscopy (TDS) and Nuclear Reaction Analysis, whereby we 
assume homogeneous impurity doping during initial boron coating
and homogeneous removal or negligible erosion of the coating, as such 
we may draw direct conclusions regarding the wall conditioning effectiveness.
\subsection{Roughness measurements}
The exposure to the plasma may give inhomogeneous sputtering which might be
of interest, as such a simple 1D profilometry measurement may be made before
and after exposure, or, if needed, a full 2D surface map. Both of the needed
devices are located in the mirror lab overseen by dr. Litnovsky.
\subsection{Thickness estimate}
Thickness estimates may be made using ellipsometry on the doped specimens 
prior to the exposure and after exposure prior to being analysed using TDS, as it is non-destructive 
, if the method is deemed adequate.
Post-mortem (i.e after exposures and TDS) the sample thickness may (also) 
be found using FIBSEM (dr. Rasinski).
\bibliographystyle{plain}
\bibliography{sources}
\end{document}
