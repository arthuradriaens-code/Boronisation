\documentclass{article}
\usepackage{amsmath}
\title{Boronisation research plan}
\author{Arthur Adriaens}

\begin{document}
\maketitle

\section{Short overview}
\begin{center}
    \addtolength{\leftskip} {-2cm} % increase (absolute) value if needed
    \addtolength{\rightskip}{-2cm}
    \begin{tabular}{||c p{4cm} p{4cm} c c||}
         \hline
         Amount & kind of samples & need & project & TOMAS days \\ [0.5ex]
         \hline\hline
         12 & hydrogen doped graphite & Test ICWC vs ECWC & W7-X & 12 \\
         6 & hydrogen doped boron-coated graphite & & & \\
         \hline
         12 & Boronized tungsten, doped with deuterium & test efficiency of IC vs EC vs GD & ITER & 12\\
         4 & Pure tungsten, doped with deuterium & & &\\
         \hline
    \end{tabular}
\end{center}
\section{Scientific relevance and reasoning}
\subsection*{W7-X}
Glow discharge in H$_2$ plasma and ECWC in He plasma have been actively used on W7-X 
to condition the wall and it's effects have been reported (\cite{WAUTERS2018235} and EUROFUSION WPS1-PR(16) 16175).
However, ECWC is only able to deposit power near the fundamental, as such the EC wall conditioning is
inhomogeneous. ICWC shouldn't have this kind of shortcoming, as a preparation to the eventual 
usage of ICWC in W7-X it should be investigated how ECWC and ICWC compare both in outgassing from
pure graphite and from boron-coated graphite (as both are possible PFMs in W7-X).
To this end baseline ECRH experiments need to be carried out on TOMAS, as well as ICWC experiments
to compare to as to draw conclusion to W7-X.
\subsection*{ITER}
It has quite recently been decided that ITER will not have a Berillium wall but a tungsten wall, 
experiments have been carried out at JET with an "ITER-like wall"\cite{DOUAI2013S1172} which
was, at the time, a Berillium wall. The author is unaware of any published wall conditioning experiments carried out 
on tungsten walls, as such the proposal is to compare all 3 ITER available wall conditioning schemes using
boronized tungsten, as most of the PFM will be for H-mode operation in ITER. As well as pure tungsten experiments
to gain information on de-trapping of exposed tungsten.
\section{TOMAS setup and days estimate}
\subsection{W7-X}
As mentioned in EUROFUSION WPS1-PR(16) 16175, ECWC outgassing follows a time law, this law 
needs to be measured on TOMAS to show that it is equiped to compare IC and EC, after which
the same kind of experiment needs to be carried out to determine the IC time law.
To get a W7-X relevant condition, the IC frequency will be set to 38MHz, mimicking the minority heating
and the second harmonic heating scenarios. The power level will be kept as high as possible whilst the electron density
will be kept to the one observed at W7-X (reflectometer measurements), the plasma species will be Helium to have
a more distinct signal from the removed hydrogen.

The exposures will all keep these same settings and vary in time to make a time-outgassing relation possible.
6 samples will be used to do the EC discharges, grouped as 2 each to make error estimation possible. After which
6 samples will be used to do the IC discharges. 

As W7-X also boronizes their walls, IC experiments will be also be carried out on 6 boronized graphite samples,
as the EC-IC relation should already be established from previous experiments, this should give additional insight
into the de-trapping efficiency of boron.
\subsection{ITER}

\section{Thickness and density estimation}
We wish to know how the various wall conditioning schemes affect the boronized PFM, to
this end we need to know both the thickness of the boron layer before and after the
exposures, as well as it's density.  We want to know the density as to see how
it is dependent on the substrate and to infer how many atoms got sputtered.
\subsection{Boronized Tungsten}
My proposal for measuring the thickness
of the boron layer is to use three methods, the first two methods concern a
general estimate of the thickness whilst the third would be sample-specific.
We first half-coat one sample by applying sticky tape during coating and
peeling it off afterwards, enabling the first method: the use of a profilometer
to determine the jump in height. The second method which we'll also apply on
this half-coated sample is the use of a SEM either by drilling small holes in
the coated surface and determine the thickness that way or looking at the side
of the sample.  The third method is to use the ellipsometer in the mirror lab,
this last method is also what we'll be using for the normal samples, it is
however good to have many thickness estimates of the half-coated sample as
we'll need it later and as it's a good check to see if the ellipsometry
measurements are accurate.  Note that most of these techniques (especially the
ellipsometry) will only be possible if the tungsten surface is polished properly.
 \vspace{0.2cm} \\
To now measure the density we'll perform ion beam analysis on the half coated
sample. To see how this works, please allow the following train of thought: After
analysing the experimental data from the IBA (ERDA) we'd know that there are x amount
of $B^{10}$ atoms per cm$^2$ and y amount of $B^{11}$ atoms per cm$^2$ on the
surface of the sample, using the mass of both of these isotopes and the
measured thickness (of which we have 3 independent measurements), we can infer
the density:
\begin{equation}
    \rho = \frac{1}{d \times 1 cm^2} \left( x \times \text{massa }B^{10}   \text{ isotope} + y \times \text{massa }B^{11}   \text{ isotope}\right)
\end{equation}
Assuming this density to hold for the other samples, we can infer how many atoms 
were sputtered from the change in thickness which we'll measure using ellipsometry.
\subsection{Boronized Graphite}
My proposal for measuring the thickness of the boron layer on graphite would be
to use two methods, we'll also be using the half-coated sample enabling the use
of a profilometer. The second method which we'll apply in conjunction is by
using the ellipsometer in the mirror lab, note that the SEM wouldn't be able to
see the difference between carbon and boron and is thus not possible to use
here. We then proceed as mentioned for tungsten, inferring the density using
IBA.
\section{Doping estimation}
As we'd later want to measure the outgassing efficiency of the various wall conditioning 
systems, we'd like to dope the samples with some atoms (e.g deuterium) and measure the concentration
before and after the various wall conditioning schemes.
We'll probably measure these concentrations using ERDA.
\section{Exposure: Erosion rate}
We'd like to expose the samples to either hydrogen, helium or mixed hydrogen
and helium. Each under different wall conditioning regimes, either Glow Discharge,
Ion Cyclotron Wall Conditioning, Electron Cyclotron Wall Conditioning or ICWC and ECWC
at the same time. We'll first do some spectroscopy to see the amount of impurities in
our plasma (if any are visible, our spectrometer is quite low-resolution). And then
expose the pure boron to test the erosion rate with different powers.
\subsection{ICWC}
In TOMAS the IC creates both neutrals and ions, mostly with energies below
1keV.  The frequency at which we may couple is still uncertain, ideally we'd like to
go as high as possible which would be a 50MHz plasma, but for now we were only able to go up to 42MHz. 
As magnetic field we will use 0.114T (2000A input current) and power-wise we'd
like to do a ramp, with values 1500W, 3500W and 5500W of injected power. 
Limiting the pressure to $10^{-4}$ mbar during the discharge of the
1500W, maintining the same base pressure (not the neutrals pressure) for the
higher powers. I.e when the penning gauge indicates $10^{-3}$ mbar without IC
and $10^{-4}$ mbar with IC at 1500W, we'll be doing 5500W for the same
$10^{-3}$ mbar gas (however this system is still under consideration). 
Whilst the powers we'll use are very small compared to
larger devices, due to the way everything is measured, as will be mentioned, it
might be possible to extrapolate to bigger devices.  ICWC at TOMAS is a
monopole working in a mode conversion scheme, as such most of the particles
evenly spread to the wall.  This enables us to extrapolate measurements
performed by the RFEA (retarding field energy analyser,able to  measure the ion distribution) 
and ToF-NPA (Time of Flight Neutral Particle energy Analyser, measure neutral distribution) 
to the full vessel and use it as a prediction on how the sample will be eroded. 
Unfortunately the retarding field energy analyser (RFEA) seems to have some
hickups for the moment when doing IC discharges which we'll try to fix over the
next couple of weeks.  If it's not properly fixed when experiments are carried
out, we will have to either rely on simulations to determine the amount of ions and
their energies in the plasma or correlate the electron density and temperature to the
ion density and temperature (if at all possible) or correlate the neutrals to the ions
(e.g assuming pure charge exchange).
\subsection{ECWC}
Efficiency of ECWC for fuel removal has been less investigated in current
devices but as it will be used for conditioning in JT-60SA and W7-X, there is
an upsurge in interest. We don't expect any 10eV$<$ energy neutrals, as such
the main player of outgassing and erosion will be ions.  As mentioned we either
need some kind of RFEA measurements to estimate the erosion rate/simulate or somehow
correlate the electron density and temperature to the ion density and
temperature.
\subsection{ECWC+ICWC}
These sometimes are used in conjunction, so we'll also do exposures with these,
note that the erosion rate will drastically increase. The max combined power
should be around 11kW of injected power, we may scale as: 1.5kW EC + 1.5kW IC,
3.5kW EC + 3.5kW IC, 5.5kW EC + 5.5kW IC, or total = 3kW, 7kW and 11kW of
power. We have tested 10kW already so this all seems possible.  I'd do the
combined erosion after the IC and EC as we'll have a more concrete idea of what
to expect.
\subsection{Glow Discharge}
Even though GD is falling out of favour due to superconducting magnets not
being easy to turn off, we'll do some exposures if time permits, with the same
pre-measurements as IC.
\section{Extrapolation to larger devices}
If we can simulate the particle flux, using e.g a modified version of the
tomator code, and the thus implied erosion using a BCA (Binary Collision
Approximation) code such as rustBCA, we might be able to verify the erosion
prediction and have confidence that this kind of code may work on
bigger devices.

It is also possible that some mechanisms are dominant and easily scalable, for
example, it might be that the IC sputtering is mainly caused by the RF sheath
formed in front of the antenna.  The RF sheath induced sputtering may scale
logaritmically with power, if this is also the case with the sample erosion, we
can be confident that it's the main sputtering candidate and thus easily
scalable to other antenna's (by simulating their sheath).

\section{Exposure: Outgassing efficiency}
This is for much later, after all the erosion estimations have been done, but
under all the previously mentioned techniques, we'll use the same measurements
and estimate the outgassing rate by constructing a sample with the same amount
of doping in the BCA simulation.

\bibliographystyle{plain}
\bibliography{sources}
\end{document}
