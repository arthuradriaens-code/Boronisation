\documentclass{article}
\usepackage{amsmath}
\usepackage{amssymb}
\title{Predicting Wall Conditioning}
\author{Arthur Adriaens}

\begin{document}
\maketitle
\section*{\textit{Theory}}
When a material has been subjected to impurities, these impurities will be
stored in interstitial lattice sites  (LS). The base material atoms are bound by a
certain energy $E_b$ whilst the trapped impurity is bound by $E_t$, as most
often $E_t>E_b$, wall conditioning in fusion reactors is effective.  In general
wall conditioning creates low-energy charge exchange neutrals (i.e neutrals who
have gained their energy through charge exchange) who, as they are not
confined, go towards the wall. Ideally these neutrals have energies sufficient
to de-trap (sputter) the impurities but low enough energies to leave the base
material intact.  Mathematically we can express the amount of impurities
leaving the wall as:
\begin{equation}
    I(t) = \sum_j\int_E Y_{jI}(E,t)\mathfrak{F}_j(E)
\end{equation}
Whereby $Y_{Ij}(E,t)$ is the energy and time-dependent impurity sputtering rate
(this is time-dependent as the impurity concentration will decrease over time)
for incoming species j and $\mathfrak{F}_j(E)$ is the incoming particle
distribution (particles/s) for species j.  The base material sputtering rate
may be given by:
\begin{equation}
    B = \sum_j\int_E Y_{jB}(E)\mathfrak{F}_j(E)
\end{equation}
Where in the base material sputtering rate may be assumed to not be
time-dependent as the material is homogeneous.  In full the total amount of
particles leaving the wall per second is thus:
\begin{equation}
    W(t) \stackrel{\Delta}{=} I(t) + B = \sum_j \int_E \left\{ Y_{jI}(E,t) + Y_{jB}(E) \right\} \mathfrak{F}_j(E)
\end{equation}
$\mathfrak{F}_j(E)$ is a measureable quantity, for example on TOMAS the neutral
fluxes are measureable \cite{DanielNPA} as well as the ions \cite{AndreiRFEA}.
$Y_{jB}$ Is straightforward to simulate using e.g rustBCA using known
material parameters, the difficulty lies in simulating $Y_{jI}$.
\section*{\textit{Simulation input}}
To simulate $Y_{jI}$ it is necessary to create a model, this model
consists of a slab of the base material with a certain concentration
of $E_t$ bound impurities. A simple first step would be to consider
the impurities to be homogeneously distributed in the slab.
%\section*{\textit{temp}}
%As is well known, when a material is subjected to a plasma a sheath forms
%abiding by Poisson's equation
%\begin{equation}
%    \chi^{\prime\prime} = \left( 1 + \frac{2\chi}{\mathfrak{M}^2}\right)^{-\frac{1}{2}} - e^{-\chi}
%\end{equation}
%with 
%\begin{equation*}
%    \chi \stackrel{\Delta}{=} - \frac{e\phi}{KTe} \qquad \epsilon \stackrel{\Delta}{=} \frac{x}{\lambda_D} = x\left(\frac{n_0e^2}{\epsilon_0 KT_e}\right)^{1/2} \qquad \mathfrak{M} \stackrel{\Delta}{=} \frac{u_0}{(KTe/M)^{1/2}}
%\end{equation*}
\bibliographystyle{plain}
\bibliography{sources}
\end{document}
