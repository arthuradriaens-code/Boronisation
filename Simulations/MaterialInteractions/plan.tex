\documentclass{article}
\usepackage{amsmath}
\usepackage{amssymb}
\title{Predicting Wall Conditioning}
\author{Arthur Adriaens}

\begin{document}
\maketitle
\section*{\textit{Theory}}
When a material has been subjected to impurities, these impurities will be
stored in interstitial lattice sites  (LS). The base material atoms are bound
by a certain energy $E_b$ whilst the trapped impurity is bound by $E_t$, as
most often $E_t>E_b$, wall conditioning in fusion reactors is effective.  In
general wall conditioning creates low-energy charge exchange neutrals (i.e
neutrals who have gained their energy through charge exchange) who, as they are
not confined, move towards the wall. Ideally these neutrals have energies
sufficient to de-trap (sputter) the impurities but are still below the energy
necessary to sputter the base material.  Mathematically we can express the
amount of impurities leaving the wall per area ($m^2$) as a functional of the form:
\begin{equation}
    \mathcal{I}[n] = \sum_j\int_E \int_\theta Y_{jI}(n(t,\mathbf{r}),E,\theta)\mathfrak{F}_j(E,\theta)
\end{equation}
Whereby $Y_{Ij}(n(t,\mathbf{r}),E,\theta)$ is the impurity concentration-, energy- and
angle of incidence-dependent impurity sputtering rate (i.e \#out/\#in) for
incoming species j and $\mathfrak{F}_j(E,\theta)$ is the incoming particle
distribution ($\frac{\text{particles}}{m^2s}$) for species j.  The base
material sputtering rate may be given by:
\begin{equation}
    B = \sum_j\int_\theta \int_E Y_{jB}(E,\theta)\mathfrak{F}_j(E,\theta)
\end{equation}
In full, the total amount of particles leaving the wall per second per unit area
is thus:
\begin{equation}
    W(t) \stackrel{\Delta}{=} \mathcal{I}[n] + B = \sum_j \int_\theta \int_E \left\{ Y_{jI}(n(t,\mathbf{r}),E,\theta) + Y_{jB}(E,\theta) \right\} \mathfrak{F}_j(E,\theta)
\end{equation}
\newpage
\section*{\textit{Simulation: homogeneous doping and erosion}} 
To simulate $Y_{jI}$ it is necessary to create a model, the chosen model is 1D
and consists of a slab of the base material with a certain concentration of
$E_t$ bound impurities. A simple first step would be to consider the impurities
to be homogeneously distributed in the slab and to erode homogeneously.
The amount of impurities in a slab of area A and depth D prior to any 
wall conditioning may be given by
$N_0 = n_0\times A \times D$, after one timestep $\Delta t$ the amount will have been reduced
by $\sum_jY_{jI}(n(t),E,\theta)\mathfrak{F}_j(E,\theta)\times A\times\Delta t\stackrel{\Delta}{=}\mathcal{I}[n(t)]A\Delta t $ 
whereby $\mathfrak{F}_j(E,\theta)$ is the amount of particles of species j per unit area per second.
Keeping E constant, a simple RFEA simulation for hydrogen impact on boron with trapped oxygen
shows a big angular dependence:
The amount of impurities in the material thus follows (if we assume homogeneous erosion):
\begin{equation}
    \frac{\text{d}N(t)}{\text{d}t} = AD\frac{\text{d}n(t)}{\text{d}t} = -A\mathcal{I}[n(t)]
\end{equation}
We thus have that the concentration of impurities in the slab changes as 
\begin{equation}
    \frac{\text{d} n}{\text{d} t} = -\frac{\bar{\mathcal{I}}[n(t)]}{D}
\end{equation}
We need to solve this equation to determine the amount of impurities sputtered per second per unit area
$\bar{\mathcal{I}}[n(t)]$, e.g forward euler we would start from a certain $n = n_0$ and change n according
to
\begin{equation}
    \Delta n = -\frac{\bar{\mathcal{I}}[n(t)]}{D}\Delta t
\end{equation}
%\section*{\textit{temp}}
%As is well known, when a material is subjected to a plasma a sheath forms
%abiding by Poisson's equation
%\begin{equation}
%    \chi^{\prime\prime} = \left( 1 + \frac{2\chi}{\mathfrak{M}^2}\right)^{-\frac{1}{2}} - e^{-\chi}
%\end{equation}
%with 
%\begin{equation*}
%    \chi \stackrel{\Delta}{=} - \frac{e\phi}{KTe} \qquad \epsilon \stackrel{\Delta}{=} \frac{x}{\lambda_D} = x\left(\frac{n_0e^2}{\epsilon_0 KT_e}\right)^{1/2} \qquad \mathfrak{M} \stackrel{\Delta}{=} \frac{u_0}{(KTe/M)^{1/2}}
%\end{equation*}
\section*{\textit{Compare with experiments}}
$\mathfrak{F}_j(E) \stackrel{\Delta}{=} \int_\theta \mathfrak{F}_j(E,\theta)$ is
a measureable quantity, for example on TOMAS the neutral fluxes are measureable
\cite{DanielNPA} as well as the ions \cite{AndreiRFEA}. As such, an angle distribution needs
to be chosen, e.g 
\begin{equation}
    \tilde{\mathfrak{F}}_j(E,\theta) \stackrel{\Delta}{=} \frac{2cos^2(\theta)}{\pi}\mathfrak{F}_j(E)
\end{equation}
$Y_{jB}$ on it's own is straightforward to simulate using e.g rustBCA using known material
parameters, the difficulty lies in simulating the compound, and thus devising $Y_{jI}$.

\bibliographystyle{plain}
\bibliography{sources}
\end{document}
