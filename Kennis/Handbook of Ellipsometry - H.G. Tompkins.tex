In the late 1800s, Drude used the phase shift induced between mutually
perpendicular components of polarized light to measure film thickness down to
tens and ones of angstroms (1 angstrom = 0.1nm).  When the mutually
perpendicular components of polarized light are out of phase, the light is said
to be "elliptically polarized" hence it became known as \textit{ellipsometry}

Elliptically polarized light has an electric field of the form
\begin{equation}
    \vec{E}(t) = \begin{bmatrix}
        E_x(t)\\
        E_y(t)
        \end{bmatrix}
        = \Re\left\{
        \begin{bmatrix}
        Xe^{i\Delta}\\
        Y
        \end{bmatrix}
        e^{i\omega(t-t_0)
            \right\}
\end{equation}
Besides the phase shift $\Delta$, the state of elliptic polarization is determined by the amplitudes X and Y. More precisely, only the \textit{relative amplitude} X/Y is relevant in ellipsometric measurements as multiplying both X and Y by a common constant merely changes the light intensity.
The relative angle can be expressed with the help of the angle $\Psi$:
\begin{equation}
    \tan \Psi  = \frac{X}{Y}
\end{equation}
Consequently, the elliptic polarization can be represented by the \textit{Jones vector}
\begin{equation}
    \begin{bmatrix}
        \sin \Psi e^{i\Delta}\\
        \cos \Psi e^{i\Delta}
    \end{bmatrix}
\end{equation}
Two states are called \textit{orthogonal} when their Jones vectors are
orthogonal in the usual sense of vector algebra.
Special cases:
\begin{itemize}
    \item Linear polarization: $\Delta = 0$ or $\pi$
    \item Circular polarization: $\Psi = \pi/4$ and $\Delta = \pi/2$ or $-\pi/2$
\end{itemize}
In the quantum-mechanical picture, the general state of elliptic polarization is a superposition
of linear polarization along x with the complex amplitude $\sin\Psi \exp i\Delta$ and the state of linear polarization along y with the complex amplitude $\cos \Psi$. Consequently the probability of
a photon passing an ideal polarizer oriented along x is $\sin^2\Psi$ and along y is $\cos^2\Psi$.

