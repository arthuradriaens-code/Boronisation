\section{General}
\subsection{Review paper: Plasma-surface interactions in tokamaks - G.M.McCRACKEN}
Two major areas of concer: 
\begin{itemize}
    \item First wall of the reactor will be damaged byb the plasma and by the reaction products
    \item Surface interactions will contaminate the plasma with impurities -> cool plasma
\end{itemize}
In many cases it is clear that the observed characteristics of the discharge
are determined more by the plasma-surface interactions than by classical plasma
physics. 

Alternative approach is to synthesize the plasma flux using ion and electron beams in order to
study the fundamental processes. This has the obvious advantage that the experiments can in general be 
simpler with easily chosen parameters, the disadvantage is that the co-operative effects, which may occur only 
when the complete plasma is present, will not be observed. A combination of both approaches is necessary.

Main purpose of limiter: protect the torus wall from damage due to violently unstable discharges and from
intense beams of runaway electrons -> usually made out of a refractory metal such as tungsten or molybdenum.
This may introduce high-Z ion contamination therefor multiple attempts of low-Z material.
Interesting: boron carbide
Three distinct modes of tokamak operation:
\begin{description}
    \item[Low-Z dominated mode]
        impurities are predominantly oxygen, carbon or nitrogen and most of the
        Ohmic power islost by radiation from these species.
        Radiation mainly comes from low ionization states at large radii.
    \item[High-Z dominated mode]
        primarily high-Z metals such as molybdenum or tungsten from the limiter.
        pre-requisite: removal of most of the adsorbed gas from the wall.
        Most lost power then originates from radiation of these elements from the center of the plasma.
        Temperature at the center is low -> hollow profile.
    \item[High-density mode]
        low impurity concentration, $Z_{eff} \approx  1$.
        Firstly the walls are cleaned, second is tha t additional working gas, hydrogen or deuterium,
        is injected during the discharge.
        This probably reduces the edge temperature, reducing sputtering and the probability of arcing on
        the surface.
\end{description}
The walls of a tokamak are bombarded by radiation, by charged particles which
have diffused across the confining magentic field, and by neutral atoms. The
first step in attempting to assess the surface phenomena is obviously to
estimate the magnitudes and energy distributions of these fluxes.  Radiation in
present-day tokamaks may account for between 20\% and 100\% of the ohmic power
input and is generally found to be fairly uniformly distributed around the
torus.  Given a minor radius r, a major radius R and Ohmic input of P, this
places an upper limit on the radiated flux to the wall of 
\begin{equation}
    \frac{P}{4\pi^2 (R-r)(R+r)}
\end{equation}
Neutral hydrogen atoms escape from the plasma as a result of resonant charge exchange
between plasma ions and incoming neutrals. Neutrals arise from gas admitted, gas desorbed 
from the walls by particle bombardment and neutralized ions which return from the wall.
-> localization of charge-exchange neutrals near the limiter and gas inlet ports.
\subsection{Carbon materials for advanced technologies}
The greatest effort in the development of fusion energy has been in the
enormously challenging area of plasma physics and plasma confinement. It is
clear that perfect containment of a fusion plasma is impossible, and that
interactions between the hot ionized plasma gas and their surroundings will
take place. In confinement systems such as the tokamak, this interaction point
is very close to, and in some cases defines, the edge of the plasma. The
components which are in line of sight of the plasma, and therefore impacted by
the hot gasses and particles, are called plasma- facing components (PFCs) or
plasma-facing materials (PFMs). The reactions between the fusion plasma and the
PFMs are severe and typically cause melting or sublimation, component
mechanical failure due to high thermal stress, and 
excessive surface erosion. The plasma ion flux and associated heat loading to
the plasma-facing materials is highly non-uniform and quite dependent on the
tokamak design.

The hot plasma gasses are made up of unburned hydrogen fuel, fusion byproducts 
such as helium, plasma electrons and impurity elements previously removed from 
PFCs, and plasma electrons. As can be seen in Eq. 1
,
 the type of particles which 
may strke the PFMs are dependent on the fusion fuel. For the D+T fuel system, 
the plasma will contain not only the D+T fuel, but high energy alpha particles (3.5 
MeV He) and neutrons (14.1 MeV). The partitioning of the reaction energy 
between helium and the neutron is both an advantage and a disadvantage for the 
D+T fuel system. Because the energetic helium nucleus quickly collides with the 
surrounding gasses, most of its energy remains in the plasma and helps to sustain 
a high plasma temperature. Conversely, the neutron has very little chance of 
collision in the low density plasma and loses its energy outside of the plasma 
(usually over meters of path length inside the structure 
of the reactor). Because less 
than 30\% of the D+T reaction energy remains in the plasma, only this fraction is 
eventually distributed on the PFCs, thus reducing the heat load handling 
requirement and material erosion. However, as discussed in Section 3 of this 
chapter, the material damage associated with the 14.1 MeV neutron collisions is 
significant and offsets the reduced D+T heat loading. 

Fusion devices can be characterized 
by how the plasma edge is defined and how 
the impacting flux and heat are handled. The classic approach is to define the 
plasma edge by placing a sacrificial 
component in contact with the plasma. This 
component, which intercepts the plasma edge particle flux, is h o w n  as a bumper 
or bumper limiter, and extends circumferentially around the torus. A second 
approach to defrning the plasma edge is to magnetically 
capture and divert the edge 
plasma onto a divertor plate well removed from the central plasma. Once the 
plasma gasses are cooled they can be pumped away. The point on the "divertor" 
where the particle flm strikes experiences 
a significant 
ion heat loadmg, and many 
techniques such as magnetic sweeping to spread the loading and puffing of gas to 
%often'' the ion impact have been used to reduce the particle flux and energy. 
Regardless of whether the limiter or divertor design is employed, the majority of 
the particle and heat flux is intercepted 
by these components (Table 1). However, 
a sigmficant flux also impacts the balance of the torus lining (generally referred to 
as the fmt wall).
\section{The boundary layer}
\subsection{Review paper: Plasma-surface interactions in tokamaks - G.M.McCRACKEN}
Boundary between a plasma and a solid surface is a complex region. A simple sheath will
normally be set up at a surface in contact with a plasma owing to charge conservation
and the high velocity of electrons with respect tot ions.
The surface will charge negatively with respect to the plasma, repelling electrons 
until the net current becomes zero.
In a thermal plasma, the sheath potential is given by
\begin{equation}
    V_s = \frac{kT_e}{2e} ln\left\{\frac{m_i (1-\Gamma)^2}{2\pi m_e}\right\}
\end{equation}
With $\Gamma$ a measure of the secondary electron emission caused by ions arriving at the surface.
The situation becomes a lot more complicated in the presence of a magnetic field, surfaces parallel
to the field lines would be expected to charge positively whilst normal surfaces would charge negatively.

The incident ions which are not backscattered will slow down to thermal velocities in the solid. When most ions
are implanted in solids, no furter movement occurs and the ion is trapped.
However, with hydrogen, in many metals there is a relatively high diffusion coefficient even near room temperature.
\section{ICWC}
\section{ECWC}
\subsection{Review paper: Plasma-surface interactions in tokamaks - G.M.McCRACKEN}
The impact of fast electrons on the vacuum wall and limiter is a well-known source of impurity release.
\subsection{Hydrogen removal by electron ... M. Fukumoto}
About Ne-ECWC

ECWC is effective for recovery from plasma disruptions by removing retained fuels in devices with carbon wall. Metal wall devices, however, He plasmas produce defects such as He bubbles in the metal plasma-facing compoents under some conditions, leading to increase in He and fuel retention and continuous release of He and fuel particles. Neon also seems to cause defects such as voids and bubles (conclusion of this paper).

Number of removed H$_2$ molecules, injected Ne and evacuated Ne atoms were evaluated by the following equations:
\begin{align}
    N^{rem}_{H2} &= \int C P_{H2} S_{H2}dt\\
    N^{inj}_{Ne} &= \int Q_{Ne}dt\\
    N^{evac}_{Ne} &= \int C P_{Ne} S_{Ne}dt
\end{align}
$C = 2.94\times 10^{20}$ [particles/Pa m$^3$] conversion factor, P's [Pa] partial pressures evaluated by the QMS, S's [m$^3$/s] the pumping speeds and Q [atoms/s] the injection rate.

Ne-ECWC increased the plasma current, advanced the start-up timing of the plasma current and decreased the line-integrated electron density.
Ne-ECWC also led to a decrease in H$_2$ recycling.


